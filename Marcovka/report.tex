\documentclass{article}
\usepackage[russian]{babel}
\usepackage[utf8]{inputenc}
\usepackage{amsmath}
\usepackage{graphicx}
\usepackage[colorinlistoftodos]{todonotes}
\usepackage{float}
\setcounter{section}{-1}
\usepackage[margin=2cm]{geometry}
\usepackage{listings}
\usepackage{multirow} 
\usepackage{caption}
\usepackage{color}
\definecolor{mygreen}{rgb}{0,0.6,0}
\definecolor{mygray}{rgb}{0.5,0.5,0.5}
\definecolor{mymauve}{rgb}{0.58,0,0.82}

\title{Отчет}

\author{Арсений Серока\\ студент группы \textnumero 4538\\Вариант \textnumero 5\\\\Пьянкова Юлия\\ студентка группы \textnumero 4539\\Вариант \textnumero 12}

\date{\today}

\begin{document}
\maketitle

\section{Формулировка Задания}
\def \n {$n = [1..4]$}
\def \Hn {$H_n(x)$}
Для марковского источника с заданной матрицей переходных вероятностей\\
найти $H(X),\ H(X|X^{\infty}),\ H_2(X),\ H_n(X)$. Построить коды Хаффмена для ансамблей.\\
Указать наилучший алгоритм кодирования для данного источника.
\newpage
\section{Вариант 5}
\subsection{Дано:}
\[
   \Phi =
  \left[ {\begin{array}{ccc}
   \frac{3}{4} & \frac{1}{4} & 0 \\[1em]
   0 & \frac{1}{4} & \frac{3}{4} \\[1em]
   \frac{1}{4} & \frac{1}{4} & \frac{1}{2} \\
  \end{array} } \right]
\]
\subsection{Решение:}
Найдем вектор $\overrightarrow{x} = (p_a,\ p_b,\ p_c)$, при этом он должен удовлетворять $\overrightarrow{x}\Phi = \overrightarrow{x}$ и $p_a + p_b + p_c = 1$

\[
 \left\{ 
   \begin{array}{ll}
       p_a(\frac{3}{4} - 1) + \frac{p_c}{4} = 0, \\[1em]
       \frac{p_a}{4} + p_b(\frac{1}{4} - 1) + \frac{p_c}{4} = 0, \\[1em]
       p_a + p_b + p_c = 1.
   \end{array} 
 \right. 
\] \\
Отсюда $\overrightarrow{x} = (\frac{3}{8}, \frac{1}{4}, \frac{3}{8})$\\
Энтропия источника:
$H(X) = -p_a\log{p_a} - p_b\log{p_b} - p_c\log{p_c} \approx 1{.}561278$. \\\\
Теперь найдём условную энтропию $H(X|X)$. По определению,\\
$
H(X|X) = \sum\limits_{y} p(y) \cdot H(X|y) =\\
= p_a \cdot H(X|a) + p_b \cdot H(X|b) + p_c \cdot H(X|c) \approx 0{.}429229296 $\\
Так как нам дана простая марковская цепь, то $H(X|X) = H(X|X^2) = \dots = H(X|X^{\infty})$\\
По свойствам энтропии:\\
$
H_2(X) = \frac{H(X^2)}{2} = \frac{H(X) + H(X|X)}{2} \approx 0{.}9952537 \\ \\
H_n(X) = \frac{H(X) + (n - 1)H(X|X)}{n} \approx \frac{1{.}561278 + (n-1)*0{.}429229}{n}
$

\begin{table}[H]
\begin{tabular}{|c|c|c|c|}
    \hline
  Символ      & a              & b              & c             \\
    \hline
  Код         & 01             & 00             & 1             \\
    \hline
\end{tabular}
\captionsetup{singlelinecheck=off,justification=raggedright}
\caption{\label{tab:widgets}Код Хаффмена для ансамбля $X$}
\end{table}

Средняя длина кода: $2 * \frac{3}{8} + 2 * \frac{1}{4} + 1 * \frac{3}{8} = 1.625$

\begin{table}[H]
\begin{tabular}{|c|c|}
aa & $(\frac{3}{8}) * (\frac{3}{4}) = \frac{9}{32}$\\\\
ab & $(\frac{3}{8}) * (\frac{1}{4}) = \frac{3}{32}$\\\\
ac & $0$\\\\
ba & $0$\\\\
bb & $(\frac{1}{4}) * (\frac{1}{4}) = \frac{1}{16}$\\\\
bc & $(\frac{1}{4}) * (\frac{3}{4}) = \frac{3}{16}$\\\\
ca & $(\frac{3}{8}) * (\frac{1}{4}) = \frac{3}{32}$\\\\
cb & $(\frac{3}{8}) * (\frac{1}{4}) = \frac{3}{32}$\\\\
cc & $(\frac{3}{8}) * (\frac{1}{2}) = \frac{3}{16}$\\\\
\end{tabular}
\captionsetup{singlelinecheck=off,justification=raggedright}
\caption{\label{tab:widgets}Вероятности пар символов}
\end{table}


\begin{table}[H]
\begin{tabular}{|c|c|}
aa & 00\\
ab & 0010\\
bb & 1010\\
bc & 110\\
ca & 001\\
cb & 101\\
cc & 11\\
\end{tabular}
\captionsetup{singlelinecheck=off,justification=raggedright}
\caption{\label{tab:widgets}Код Хаффмена для ансамбля $X^2$}
\end{table}
\\
Средняя длина кодового слова: $2 * \frac{9}{32} + 4 * \frac{3}{32} + 4 * \frac{2}{32} + 3 * \frac{6}{32} + 2 * 3 * \frac{3}{32} + 2 * \frac{6}{32} = 2{.}6875$\\
На один символ: $1{.}34375$

Оптимальный код Хаффмена для исходной матрицы: для первого символа — код Хаффмена для ансамбля $X$, для второго согласно таблице:
\begin{table}[H]
\begin{tabular}{|c|c|c|c|c|c|c|c|c|c|}
a & a & a & b & b & b & c & c & c & предыдущий символ\\
a & b & c & a & b & c & a & b & c & следующий символ\\
0 & 1 &   &   & 0 & 1 &10 &11 & 0 & код\\
\end{tabular}
\captionsetup{singlelinecheck=off,justification=raggedright}
\caption{\label{tab:widgets}}
\end{table}

Средняя длина: $1 * \frac{9}{32} + 1 * \frac{3}{32} + 1 * \frac{1}{16} + 1 * \frac{3}{16} + 2 * 2 * \frac{3}{32} + 1 * \frac{3}{16} = 1.1875$. Поэтому алгоритм кодирования есть лучший для данного источника. 
\newpage
\section{Вариант 12}
\subsection{Дано:}
\[
   \Phi =
  \left[ {\begin{array}{ccc}
   \frac{1}{4} & \frac{3}{4} & 0 \\[1em]
   0 & \frac{1}{2} & \frac{1}{2} \\[1em]
   \frac{1}{3} & \frac{1}{3} & \frac{1}{3} \\
  \end{array} } \right]
\]
\subsection{Решение:}
$\overrightarrow{x} = (\frac{4}{25}, \frac{2}{25}, \frac{9}{25})$\\
Энтропия источника:
$H(X) = -p_a\log{p_a} - p_b\log{p_b} - p_c\log{p_c} \approx 1{.}4619011$. \\\\
Теперь найдём условную энтропию $H(X|X)$. По определению,\\
$
H(X|X) = \sum\limits_{y} p(y) \cdot H(X|y) =\\
= p_a \cdot H(X|a) + p_b \cdot H(X|b) + p_c \cdot H(X|c) \approx 0{.}409413 $\\
Так как нам дана простая марковская цепь, то $H(X|X) = H(X|X^2) = \dots = H(X|X^{\infty})$\\
По свойствам энтропии:\\
$
H_2(X) = \frac{H(X^2)}{2} = \frac{H(X) + H(X|X)}{2} \approx 0{.}9356573 \\ \\
H_n(X) = \frac{H(X) + (n - 1)H(X|X)}{n} \approx \frac{1{.}461901189 + (n-1)*0{.}4094134997}{n}
$

\begin{table}[H]
\begin{tabular}{|c|c|c|c|}
    \hline
  Символ      & a              & b              & c             \\
    \hline
  Код         & 01             & 0              & 11             \\
    \hline
\end{tabular}
\captionsetup{singlelinecheck=off,justification=raggedright}
\caption{\label{tab:widgets}Код Хаффмена для ансамбля $X$}
\end{table}

Средняя длина кода: $2 * \frac{4}{25} + 1 * 1\frac{2}{25} + 2 * \frac{9}{25} = 3\frac{8}{25} = 1.52$

\begin{table}[H]
\begin{tabular}{|c|c|}
aa & $(\frac{4}{25}) * (\frac{1}{4}) = \frac{1}{25}$\\\\
ab & $(\frac{4}{25}) * (\frac{3}{4}) = \frac{3}{25}$\\\\
ac & $0$\\\\
ba & $0$\\\\
bb & $(1\frac{2}{25}) * (\frac{1}{2}) = \frac{6}{25}$\\\\
bc & $(1\frac{2}{25}) * (\frac{1}{2}) = \frac{6}{25}$\\\\
ca & $(\frac{9}{25}) * (\frac{1}{3}) = \frac{3}{25}$\\\\
cb & $(\frac{9}{25}) * (\frac{1}{3}) = \frac{3}{25}$\\\\
cc & $(\frac{9}{25}) * (\frac{1}{3}) = \frac{3}{25}$\\\\
\end{tabular}
\captionsetup{singlelinecheck=off,justification=raggedright}
\caption{\label{tab:widgets}Вероятности пар символов}
\end{table}


\begin{table}[H]
\begin{tabular}{|c|c|}
aa & 0011\\
ab & 1011\\
bb & 00\\
bc & 10\\
ca & 001\\
cb & 101\\
cc & 111\\
\end{tabular}
\captionsetup{singlelinecheck=off,justification=raggedright}
\caption{\label{tab:widgets}Код Хаффмена для ансамбля $X^2$}
\end{table}
\\

Средняя длина кодового слова: $4 * \frac{1}{25} + 4 * \frac{3}{25} + 2 * 2 * \frac{6}{25} + 3 * 3 * \frac{3}{25} = 2.68$\\
На один символ: $1.34$\\

Оптимальный код Хаффмена для исходной матрицы: для первого символа — код Хаффмена для ансамбля $X$, для второго согласно таблице:
\begin{table}[H]
\begin{tabular}{|c|c|c|c|c|c|c|c|c|c|}
a & a & a & b & b & b & c & c & c & предыдущий символ\\
a & b & c & a & b & c & a & b & c & следующий символ\\
0 & 1 &   &   & 0 & 1 & 0 &10 &11 & код\\
\end{tabular}
\captionsetup{singlelinecheck=off,justification=raggedright}
\caption{\label{tab:widgets}}
\end{table}
Средняя длина: $1 * \frac{1}{25} + 1 * \frac{3}{25} + 2 * 1 * \frac{6}{25} + 1 * \frac{3}{25} + 2 * 2 * \frac{3}{25} = 1.24$
Поэтому алгоритм кодирования тоже есть лучший для данного источника. 
\end{document}
