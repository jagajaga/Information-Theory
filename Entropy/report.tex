\documentclass{article}
\usepackage[russian]{babel}
\usepackage[utf8]{inputenc}
\usepackage{amsmath}
\usepackage{graphicx}
\usepackage[colorinlistoftodos]{todonotes}
\usepackage{float}
\setcounter{section}{-1}
\usepackage[margin=2cm]{geometry}
\usepackage{listings}
\usepackage{multirow} 
\usepackage{caption}
\usepackage{color}
\definecolor{mygreen}{rgb}{0,0.6,0}
\definecolor{mygray}{rgb}{0.5,0.5,0.5}
\definecolor{mymauve}{rgb}{0.58,0,0.82}

\lstset{ %
  backgroundcolor=\color{white},   
  basicstyle=\footnotesize,        
  breakatwhitespace=false,         
  breaklines=true,                 
  captionpos=b,                    
  commentstyle=\color{mygreen},    
  deletekeywords={...},            
  escapeinside={\%*}{*)},          
  extendedchars=true,              
  keepspaces=true,                 
  keywordstyle=\color{blue},       
  morekeywords={*,...},            
  numberstyle=\tiny\color{mygray}, 
  rulecolor=\color{black},         
  showspaces=false,                }
\title{Отчет}

\author{Арсений Серока\\ студент группы \textnumero 4538\\Вариант \textnumero 19}

\date{\today}

\begin{document}
\maketitle

\section{Формулировка Задания}
\begin{enumerate}
\def \n {$n = [1..4]$}
\def \Hn {$H_n(x)$}
\item Построить эмпирическое распределение вероятностей на последовательностях длины \n. Замечание: рассмотрите «скользящие» блоки, т.е. в последовательности длины $N$ наблюдается $N-n+1$ блоков длины $n$.
\item Оценить энтропию на букву источника \Hn, приписав нулевые вероятности последовательностям букв, которые не встретились в файле при \n. 
\item Оценить энтропию на букву источника \Hn, приписывая вероятности порядка $1 / N^n$ последовательностям букв, которые не встретились в файле и соответственно нормируя вероятности остальных последовательностей, при \n.
\item Применить к заданной последовательности один из известных архиваторов (например, RAR, pkzip). Сопоставить результаты с оценками энтропии на сообщение.
\end{enumerate}
\newpage
\section{Исходный код программы на языке haskell}

\begin{lstlisting}[language=haskell]
module Main where

import qualified Data.List              as L
import           Data.Map
import           System.Environment

countSequences :: Int -> String -> Map String Int
countSequences n = fromListWith (+) . flip zip (repeat 1) . subsequencesN n

subsequencesN :: Int -> [a] -> [[a]]
subsequencesN n xs = [ take n x | x <- L.tails xs, length x >= n ]

main :: IO ()
main = do
    [file] <- getArgs
    fileContents <- readFile file
    let inputSize = fromIntegral (length fileContents) :: Double
    mapM_ putStrLn $ do
        n <- [1.0..4.0] :: [Double]
        let sequences = toList $ countSequences (round n) fileContents
        let size      = length sequences
        let totalSeq  = inputSize - n + 1.0
        let missCnt   = fromIntegral (255^round n - size) :: Double
        let missProb  = 1.0 / (inputSize ^ round n)
        let existProb = 1.0 - missCnt * missProb
        let eNorm     = -missCnt * missProb
        let eZero     = 0.0 :: Double
        let coef      = [ (i, j) | (seq, len) <- sequences,
                    let i = (fromIntegral len :: Double) / totalSeq, 
                    let j = i * existProb ]
        let (a, b)    = Prelude.foldl (\(a,b) (c,d) -> 
                (a - c * logBase 2 c, b - d * logBase 2 d))
            (eZero, eNorm) coef
        return $ show (round n, a / n, b / n)
\end{lstlisting}
\newpage
\section{Входные Данные}
\begin{table}[H]
\begin{tabular}{|l|r|}
\hline
    Имя файла & grammar.lsp\\\hline
    Размер (байт) & 3721\\\hline
\end{tabular}
\end{table}
\subsection{Результат архивации данного файла}
\begin{table}[H]
\begin{tabular}{|l|c|r|l|}
\hline
\multicolumn{2}{|l|}{Размер} & \multirow{2}{*}{Команда} & \multirow{2}{*}{Имя Файла} \\
\cline{1-2}
Байт  & Бит & & \\
\hline
1283 & 10264 & bzip2         & grammar.lsp.bz2\\
1376 & 11008 & tar cjvf      & grammar.lsp.tar.bz2\\
1353 & 10824 & tar czvf      & grammar.lsp.tar.gz\\
1388 & 11104 & zip -r        & grammar.lsp.zip\\
1234 & 9872  & gzip -c -9 -n & grammar.lsp.gz \\ \hline
\end{tabular}
\end{table}

\section{Результаты}
\begin{table}[H]
\begin{tabular}{|c|c|c|}
\hline
n & $P = 0$ & $P = 1 / N^n$\\
\hline
1 & 4.632267666454032 & 4.42903009900037\\
2 & 3.7186185597260906 & 3.702301958586279\\
3 & 2.9078214400980222 & 2.906933101640151\\
4 & 2.348285677432823 & 2.3482363251277403\\
\hline
\end{tabular}
\captionsetup{singlelinecheck=off,justification=raggedright}
\caption{\label{tab:widgets}Энтропия на байт источника}
\end{table}
\begin{table}[H]
\begin{tabular}{|c|c|c|}
\hline
n & $P = 0$ & $P = 1 / N^n$\\
\hline
1 & 17236.667986875455 & 16480.420998380374\\
2 & 13836.979660740782 & 13776.265587899545\\
3 & 10820.003578604741 & 10816.698071203002\\
4 & 8737.971005727533  & 8737.787365800321 \\
\hline
\end{tabular}
\captionsetup{singlelinecheck=off,justification=raggedright}
\caption{\label{tab:widgets}Энтропия файла}
\end{table}
\subsection{Вывод}
Как мы видим, энтропия файла при $n = 4$ в $1.13$ раз меньше, чем размер файла при лучшем сжатии. Это можно объяснить тем, что на маленьком файле архиватор не может достичь энтропии по разным причинам. Например, если архиватор использует алгоритм Хаффмана для сжатия, то надо передавать таблицу, что дает лишний объем.
\end{document}
